
\chapter{Getting Started Using \program}

\section{Parsing Philosophy}
\label{sec:philosophy}
\index{parsing, philosophy}

\program\ does not think in terms of lexical structures (e.g. lines, 
paragraphs, pages, etc.), instead, the document is concieved of
as a ``chunk'' of text which may be subdivided into a number of
sub-chunks (which may themselves be further sub-divided, and so on). 
Any amount of text may be a chunk (1 or more characters) and
\program\ just views a chunk as comprising a string of characters (which
may be impregnated with formatting characters like $\backslash$n 
and $\backslash$r).

Parsing rules may either divide the current ``working chunk'' 
and pass it on to child rules or 
it may work on the current working chunk by 
matching part or all of the text chunk and doing something with
it (and we'll discuss the things you can do with matched text later). 
If the current rule can't fully use the matched text it will try
to pass the remaining chunk of text to a sibling rule.
If no siblings exist, then that portion of the text chunk is 
errored out by the program.\index{parsing, error levels}

There are different levels of errors
in \program\ and these are presented in order of ``seriousness'' in 
table \ref{tab:errors}. It can sometimes be helpful to run a tight ship
and catch all errors when you are trying to debug your rules. On a
production run, where input file formats may vary from file to file, 
upping the error threshold is more useful so you can get the job finished
and treat the errors for each file individually later. 

{\footnotesize
\begin{table}
 \begin{center}
 \caption{Defined Error Levels in \program .}
 \index{parsing, error levels}
 \label{tab:errors}
 \index{ \program , GUI Mouse bindings }
  \vskip 12pt
  \begin{tabular}{|llll|} \hline
Error Level & Error Name & Description & Example Error \\ \hline\hline
 0 & Trivial & Unlikely to be missed information. & Unmatched Whitespace  \\
 1 & Normal  & Likely to be missed information. & Unmatched Character String \\
 2 & Serious & Missing/mallformed Parser rule. & non-tag $<$match$>$ rule missing \\
   & & & child rules. \\
 99 & Don't Halt & No errors will stop this baby. & | \\ \hline 
   \end{tabular}
  \end{center}
\end{table}
}

\section{Running the program in GUI mode} 
\label{sec:gui}
\index{ \program , GUI }

The GUI mode presents the user with an interactive interface with which to 
see the opperation of the parser as it goes through the input text.
There are short cut mouse and key bindings that will help to accelerate 
the work of the user in debugging and writing their rules.

To fire up \program\ in the GUI mode:

From a UNIX shell type:
\begin{quote}
{\tt $>$ txt2XML.pl -g }
\end{quote}

From Windows: click on the icon called ``gtxt2XML''.
 
\subsection{Orientation}
\index{GUI, orientation }

You will notice that there are
several text areas, and a menu/command button bar (refer to figure
\ref{fig:txt2XML_in_action}). I'll give a quick orientaion on each
of these below, you may wish to skip over this section to the short
tutorial which follows it (section \ref{sec:guitutorial}).

\subsubsection{Command Buttons/Menu Bar}
In one form or another the menu/command button bar contains all of
the options and commands you will need to run \program .

\subsubsubsection{Command Buttons}
\index{GUI, command buttons}

The command buttons available are
the ``\runbutton '', ``\forward1rulebutton '', ``\back1rulebutton '', 
``\nexthaltbutton '' and ``\toggleinputchunkviewbutton ''. 
There is a ``Parsing Score'' display on the far right
of the command button bar.

Pressing the \runbutton\ causes the application to re-run the parser on the
text which is currently in the \inputtextarea\ using the parsing rules which
are currently in the Rules Text Area. Output from the parser run is then 
shown in the Output Text Area. 

Pressing the \forward1rulebutton\ will make the application advance forward
1 rule if it can. You will see the current rule highlighted in 
\currentrulehighlightcolor\ 
in the Rules Text Area shift down 1 rule and the Input Chunk highlighted in
\inputchunkhighlightcolor\ 
mark the chunk of text that was fed into that rule.

Pressing the \back1rulebutton\ causes the application to return to the 
previous rule, if one exists. As for the \forward1rulebutton\ 
the highlighting in the \inputtextarea\ will change
to indicate the working chunk for the current rule.

Pressing the \toggleinputchunkviewbutton\ will toggle the colorization
of the working chunk. The default is to colorize the working chunk
with a single color: \inputchunkhighlightcolor . In this mode the 
color of the button will be yellow. The other mode is to 
colorize the start, matching, and ending portions of the working chunk. 
The button will appear pink in this mode.

\subsubsubsection{Options Menu} 
\index{GUI, options menu}

The Options Menu contains the following options: 
``\changetooldisplaysizeoption '', ``\changetextfontsizeoption '',
``\changetextfgbgcoloroption '', and the ``\quitoption '' entries.

Selection of the \changetooldisplaysizeoption\ entry will allow the user
to pick from a limited menu of other display sizes. 
Numbers in parenthesis next to the entries are guides to the
user of the general screen size in pixels of the given display.
This selection will change the point size of the font too.

Selection \changetextfontsizeoption\ entry will allow the user to 
change the point size of the font of the tool without changing
the overall display.

Selection of the \changetextfgbgcoloroption\ entry allows the user to 
change the display colors of the text areas within the tool.

Selection of the \quitoption\ entry exits the program. You will be
NOT asked if you want to save any edited files (you have been warned!).

\subsubsubsection{Parsing Control Menu} 
\index{GUI, parsing control menu}

The Parsing Control Menu contains the following options:
``\haltparseronerrorleveloption'', ``\taggederrorsinoutputxmltextoption'',
``\entifycharsinoutputxmloption'', and the ``\allownullmatchesinoutputxmltextoption'' 
entries.

Selection of the \haltparseronerrorleveloption\ allows the user to change
the level of error (or higher) that will stop the program. If ``normal''
error level is selected then ``trivial'' errors will be noted, but passed by
while any ``normal'' or ``serious'' error the parser encounters will halt the program. 
Setting to ``Don't Halt'' prevents the program from halting on any error 
once it starts its parsing run. 
Refer to table \ref{tab:errors} to see the defined error levels. 

Selection of the \taggederrorsinoutputxmltextoption\ entry allows the user
to select whether or not errors encountered in the parsing run will show
up in the output text (the errors are tagged with the string ``ERROR'').
Regardless of this setting all errors still show up in the \errortextarea . 

Selection of the \entifycharsinoutputxmloption\ changes whether or not 
characters within tags in the output text are entified.  
There are a number of reasons for toggling the entification. We have 
found cases where either turning entification on/off improves readablity 
of the output text.

Selection of the \allownullmatchesinoutputxmltextoption\ entry is a 
\index{GUI, \allownullmatchesinoutputxmltextoption }
special use option. It allows one to tag `empty' text. A case of this 
is when one might want to record the existence of an empty tag in SGML.
Normally, it is safe (and desirable) to set this option to ``No''. 

\subsubsubsection{Help Menu} 
\index{GUI, help menu}

The Help Menu contains the following options:
``\abouthelpoption'', ``\parsinghelpoption'',
and the ``\knownbugsoption'' entries.
Selecting each of these entries will bring up a short help message on that
topic. The \knownbugsoption\ is notoriously out of date.
If you find a bug, please be sure to let us know about it! (send a bug
report to \bugmail\ please!);

\subsubsubsection{Parser Score Display}
\index{GUI, parser score display}

The ``score'' from parsing the text is displayed here. The definition 
of the score is 
\begin{equation}
score = \frac{Number of characters of text that were successfully parsed}{Number of characters of text which exist in the input document}
\index{parser, score definition}
\end{equation}

\subsubsection{\errortextarea }

The \errortextarea\ is where all of the accumulated errors
from a parsing run are displayed. Every time the \runbutton\ is pressed this
area is cleared out for a fresh batch of errors (fun!).

Error messages are only loosely formated in the log. A typical message
would appear as:
\begin{quote}
{\tt  ERROR:[number:0][level:1] Zhao J.-L., Tian K.P. }
\end{quote}
where ``number:'' records the order in which the error occured,
``level:'' provides the numerical representation of the error level
(see table \ref{tab:errors}) and the remaining text is the chunk of 
text which was errored out. 

\subsubsection{\inputtextarea }
\index{GUI, \inputtextarea }
\label{sec:inputtextarea}

The input ASCII text is shown here. The text shown will be 
highlighted depending on the setting of the 
\toggleinputchunkviewbutton . In the default highlighting scheme
the working chunk appears highlighted in \inputchunkhighlightcolor\ and
the rest of the text plain. In the second mode, the working chunk highlighting
is more complex: the start, matching, and ending portions of the working chunk
are all colored differently.

You can edit the text in the \inputtextarea . The editing process is simple,
\index{GUI, editing input text}
position the mouse over a section of text and click the left mouse button
to relocate the text cursor. Alternatively, you can use the arrow keys to 
move the cursor. Now, just type in new characters on the keyboard to add
text. You can ``swipe'' regions of text with the mouse by holding down the
left mouse button. Swiped text will appear with a grey background. If you 
hit the {\tt backspace} or {\tt delete} keys the swiped text will be deleted
from the \inputtextarea . Now, you can relocate text by swiping it first
then clicking the middle (UNIX) or right (Windows) mouse buttons. 
Swipped text will be duplicated starting from the spot over which you 
have positioned the mouse. 
Unfortunately, there is no buffer history. One day we'll replace this simple
editor with something more heavy-weight.

You can set a break point by positioning the mouse over the point in the  
\index{GUI, setting input break point}
input text you wish to stop feeding the parser text and pressing the 
$<$cntl$>$ button and then left-clicking the mouse.
The breakpoint character will appear as red on a blue cell. To clear this
breakpoint, press $<$alt$>$-c.

There are three green buttons: ``Load File'', ``Save'', and ``Save As'' 
which appear at the top of the \inputtextarea .
Use these buttons to bring in new input files, overwrite your editted text
to a file (defaults to the name of the original file),
or save your buffer to a new file.

\subsubsection{\rulestextarea }
\index{GUI, \rulestextarea }
\label{sec:rulestextarea}

The \rulestextarea\ displays the parsing rules which are an XML document.
The various rules within body of the document are colorized to ease
reading of the rules.  The ``current rule'' is displayed in this area
and it will be highlighted in \currentrulecolor . At the end of a parsing
run, the {\em last} rule successfully parsed with will be the current rule.
You can advance/go back to the next/prior rule by clicking on the 
\forward1rulebutton / \back1rulebutton\ or by using 
the $<$alt$>$-f/$<$alt$>$-b keypresses.  

You can hop to any rule of interest that was successfully parsed
by pressing down the $<$cntl$>$ button and left clicking the mouse 
on the rule. 
The current rule will become the rule you clicked on 
and will be highlighted in \clickhighlightrulecolor .
Note that if the new current rule contains 
children, than for this particular function both the leading element and 
the closing element will be
highlighted to improve readability.


The \rulestextarea\ has all of the same editing abilities as the 
\inputtextarea . When you type in new rules the text will be normal. 
When you run the parser, if the newly added rules are valid XML, they
will be colorized. The failure of \program\ to colorize your new
rules will be printed to STDERR\footnote{we haven't found a way to capture
this information into the GUI yet. If you know a way, send me an email!
-b.t. \bugmail} and detail the location in your rules where there is 
a problem with the XML format.
The rules themselves are discussed in section \ref{sec:rules_def}. 

As for the \inputtextarea\ there are three green buttons at the top
of the widget: ``Load File'', ``Save'', and ``Save As''. These have the
same meanings and abilities.

\subsubsection{\outputtextarea }
\index{GUI, \outputtextarea }

The \outputtextarea\ displays output XML text in a colorized format.
Elements are colored \outputelementcolor\ and error elements are
colored \outputerrorcolor . Tagged text is left normal.

The \outputtextarea\ is not editable and only has two of the three
green file function buttons that the \inputtextarea\ and 
\rulestextarea\ have. You can save / save as your output file.
As for the other two widgets, the ``Save'' and ``Save As'' buttons 
have the same meanings.

\subsection{Short Tutorial}
\label{sec:guitutorial}
\index{ \program , GUI tutorial}

Alright, lets take this baby out for a spin, shall we?
Fire up the GUI and load in the sample input ASCII text file and 
rules file we provide in our distribution.

\section{Command line/batch mode} 
\label{sec:commandline}\index{ \program , command line mode}
\index{ \program , batch mode}

Most of the time you dont want the GUI. You want to quietly parse
a pile of files without any fireworks. Command line mode is for you.

\begin{quote}
perl.exe \program .pl 
\end{quote}



