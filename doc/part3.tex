
\chapter{About \program\ Rules}

\section{Introduction to the Parsing Rules} 
\label{sec:rules_def}
\index{ rules, introduction}

Introduction to rules. They are XML document. 
Format of a rules document (how to nest, indent).
\index{ rules, document format }
Rules have attributes (see table for allowed attributes).
\index{ rules, attributes }
Regex matching is Perl matching.  

{\small
\begin{table}
 \begin{center}
 \caption{Available Parsing Rules.}
 \label{tab:parsing_rules}
 \index{ rules, list of}
  \vskip 12pt
  \begin{tabular}{|lll|} \hline
Parsing Rule & Allowed Attributes & Description \\ \hline\hline
 & & \\
txt2XML & roottag$^{1}$ & Root element in XML rules document. \\ 
 & & Doesn't have to have child rules, but \\
 & & without them not much of anything will \\
 & & get parsed! \\
 & & \\
match & start, end, tag &  Tries to match text passed to it \\
 & statusOfStart, & from the current working chunk. If \\
 & statusOfEnd, & successful the matched text is either tagged\\
 & test & and inserted into the output document or passed \\
 & & to child rules (if they exist). \\
 & & \\
repeat & | & This rule will cause to loop over its  \\
 & & child rules until {\em none} of the child \\
 & & rules is successful in extracting something \\
 & & from the current chunk. Repeat rules \\
 & & must have child rules. \\
 & & \\
choose$^{2}$ & | & \\
 & & Must have child rules. \\
 & & \\
halt & | & Halts the parser run when encountered. \\
 & & In the GUI, it is possible to proceed from one \\
 & & halt rule to the next. Similar in functionality \\
 & & to break point functionality in C debuggers. \\
 & & \\
ignore & start, end, tag & Tries to match text passed to it\\
 & statusOfStart, & from the current working chunk. \\
 & statusOfEnd, & If successfull it will ignore this text \\
 & test & (so it won't error out or appear in output) \\
 & & Ignore rules may not have children. \\
 & & \\ \hline 
   \end{tabular}
  \end{center}
$^{1}$This attribute is required. \\ 
$^{2}$Use of this rule is deprecated.
\end{table}
}

\section{Examples of Rules Use} 

