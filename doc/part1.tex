
\chapter{Introduction}
\label{introduction}

\section{What is \program ?}
\index{ \program , description}

\program\ is a Perl/Tk program designed to help with the parsing of arbitrarily
formatted ASCII text into XML. To accomplish this goal the
program is run with a user-specified set of parsing rules that direct
\program\ on how to tag/markup the input text.
This program has been tested on both UNIX and Windows 95/98 computers
but actually has a fair chance of running on any computer that has
Perl installed and a C compiler (C compiler NOT needed for Windows 
versions to run; see section \ref{sec:installing} for software requirements).

% FIGURE of txt2XML in action
\begin{figure}
\caption{\program\ in action.}
\label{fig:txt2XML_in_action}
 \mongovarybothside txt2XML_in_action.ps 5.5in 5.5in
In this figure \program\ is shown running in the sliding frames mode
(the default display mode when the Perl Tk::SplitFrame module is 
installed).
\index{ \program , description}
\end{figure}

\program\ may be run as a command line call or with its graphical user
interface (GUI, figure \ref{fig:txt2XML_in_action}). 
The GUI features a WYSIWYG interface designed to ease the 
writing and debugging of parsing rules. In the graphical mode the user may 
interactively run, debug, edit the input or rule files, and then re-run 
the parser any number of times. 
Keyboard and mouse shortcuts allow the user to jump to any point in
the parsing run to examine exactly what chunk of text was fed to a 
rule, and how the rule reacted to it. 
From the GUI, the user may save/load new output text, input and rules text 
at any point. 
 
\section{Installing \program }
\label{sec:installing}

\section{Software Requirements}
\index{ \program , requirements}

You will firstly need a copy of the \program\ distribution archive.
These are obtainable from our anonymous FTP server 
{\tt adc.gsfc.nasa.gov} in $\backslash$pub$\backslash$?? or off of
the WWW from our page at {\tt http://adc.gsfc.nasa.gov/adc/adc\_software.html}. 

You will need an installation of Perl, version 5.00503 or better.
On many UNIX systems Perl is probably already installed. You can 
check its version using:
\begin{quote}
$>$ perl -v
\end{quote}
or from within a MS-DOS command shell:
\begin{quote}
$>$ perl.exe -V
\end{quote}
\index{Perl}
If you find that you need to upgrade Perl you may obtain a copy from 
{\tt ftp.cpan.org}. RedHat Linux users may wish to just obtain
the binary rpm from {\tt ftp.redhat.com} (don't forget to check the
signature of the rpm file!). Windows users can obtain
a free copy of ActivePerl from ActiveState at 
\activestatewebsite\
(obtain build 521 or better).  

\begin{table}
 \begin{center}
 \caption{Perl Modules Required to Run \program .}
 \label{tab:perl_modules}
 \index{Perl, modules}
  \vskip 12pt
  \begin{tabular}{|llll|} \hline
 Perl & Required & \multicolumn{2}{c|}{Associated Files by FTP site} \\
Module & Version & CPAN (UNIX/W95)$^{1}$ & ActiveState (W95)$^{2}$ \\ \hline\hline
 & & & \\
XML::Parser & 2.27+ & XML-Parser-2.27.tar.gz & Included in ActivePerl \\
 & & & \\
XML::DOM & 1.25+ & XML-DOM-1.25.tar.gz & XML-DOM.zip \\
 & & & \\
Tk & 8.00.013+ & Tk800.013.tar.gz & Tk.zip \\ 
 & & & \\
Tk::SplitFrame$^{3}$ & 0.01+  & Tk-DKW-0.01.tar.gz & | \\ 
 & 1.04+ & Tk-GBARR-1.0401.tar.gz & \\
 & 0.07+ & Tk-Contrib-0.07.tar.gz & \\
 & & & \\ \hline 
   \end{tabular}
  \end{center}
$^{1}$Location: {\tt ftp://ftp.cpan.org/pub/CPAN/modules/by-module} \\ 
$^{2}$Location: {\tt http://www.activestate.com/packages/zips} \\ 
$^{3}$Module is optional. Can be difficult to install on W95 platforms.
\end{table}

\index{Perl, modules}
Depending on the distribution of Perl you installed you will need to 
obtain and install the Perl modules in table \ref{tab:perl_modules}: 
Tk::SplitFrame is an optional module, if it is installed the user
can select to display the tool using a sliding frames mode (this 
is the mode used in figure \ref{fig:txt2XML_in_action}).
 Otherwise, \program\ will only display
with all of the text areas split into separate windows.

For Windows users it is definitely far easier to go the ActivePerl
route and install the PPD (module) files they provide. 
You can obtain the ActiveState PPD files (archived in zip format) from
{\tt www.activestate.com/packages/zips}. 

To use the CPAN modules (archived as gzip'd tarballs) 
UNIX and Windows users will need a decent C/C++ compiler (gcc 2.8.1 works
for us) to build the 
XML::Parser module and a compatible {\tt make} utility ({\tt NMAKE} 
works for us on W95 platforms). CPAN modules are available at 
{\tt ftp.cpan.org/pub/CPAN/modules/by-module}.

\subsection{Installation on UNIX systems}
\index{ \program , UNIX installation}

You will first need to insure that 
you have the correct version of Perl and all of the needed Perl
modules installed.  

To install a Perl module, unpack the tarball:
\begin{quote}
{\tt $>$ gunzip -c file.tar.gz $|$ tar xvf - }
\end{quote}
Change into the new sub-directory and then type:
\begin{quote}
{\tt $>$ perl Makefile.PL; make ; make test; make install}
\end{quote}

You will need to do this for each module in turn. Installation of
the modules will go easier if you install them in the order:
\begin{quote}
XML-Parser-2.27.tar.gz \\
XML-DOM-1.25.tar.gz \\
Tk800.013.tar.gz \\
Tk-GBARR-1.0401.tar.gz  \\
Tk-Contrib-0.07.tar.gz  \\
Tk-DKW-0.01.tar.gz 
\end{quote}

Now unpack the \program\ tarball \program \_latest.tar.gz:
\begin{quote}
{\tt $>$ gunzip -c \program \_latest.tar.gz $|$ tar xvf - }
\end{quote}

You can use \program\ right from the new sub-directory. There is
a simple install script, {\tt install.sh}, which you may use to
install \program\ in a favorite directory. If 
you can get super-user access (if you don't know what this is
you don't have it) you can install the program in a system
directory using the install script.

\subsection{Installation for users with W95 platforms}
\index{ \program , W95 installation}

Installation on Windows platforms is even easier than for
UNIX, however, just as in the UNIX install you will first 
need to insure that you have the correct version of Perl and 
all of the needed Perl modules installed.
If you need to install/upgrade Perl, download the zip file from 
the ActiveState web site \activestatewebsite\ and run the
installer. The modules are similarly easy to install. Download
the zip files, unpack them, and type (within MS-DOS shell):
\begin{quote}
{\tt $>$ ppm install XML-DOM.ppd}
\end{quote}
to install the XML-DOM module. Just substitute ``XML-DOM.ppd'' for 
the appropriate file name of the other modules.

Now unpack the latest \program\ version zip file. 
You can now invoke the program from the spot you unpacked it.

\section{License}
\label{sec:license}\index{ \program , license}

Will \program\ be released as open software? Is GPL or BSD license appropriate?
Does NASA have its own license for software?

Please read the License file which is included in the archive distribution.

\section{Support/Contact/Bug Reports}
\index{Bug reports}\index{Support}
\label{contact}

There is no official user support for \program . We are however 
sympathetic to the first time user who has failed to get the program 
running even after having {\it read the documentation}.
Please send requests for help to:
\begin{quote}
\helpmail\
\end{quote}
Remember we are busy on another project or two, so our response may 
take some time.  

Comments on how we may work to improve this program and bug reports 
(especially patches!) are greatly appreciated. 
Please send email to:
\begin{quote}
\bugmail\
\end{quote}

%Brian Thomas {\it brian.thomas.1@gsfc.nasa.gov} \\ 
%Ed Shaya {\it edward.shaya.1@gsfc.nasa.gov} \\ 
%Brian Holmes {\it brian.holmes.1@gsfc.nasa.gov} \\ 

\section{Credits}
\index{Credits} 

 \begin{center}
  \begin{tabular}{ll} 
 Concept:         & Ed Shaya, Brian Thomas, Brian Holmes. \\ 
 Parser Design:   & Brian Thomas, Ed Shaya, and Jim Blackwell. \\ 
 GUI Design:      & Brian Thomas, Ed Shaya and Brian Holmes. \\ 
 Code Written by: & Brian Thomas and Ed Shaya. \\ 
  \end{tabular}
 \end{center}

